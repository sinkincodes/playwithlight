%% LyX 2.3.4.2 created this file.  For more info, see http://www.lyx.org/.
%% Do not edit unless you really know what you are doing.
\documentclass[english]{article}
\usepackage[LGR,T1]{fontenc}
\usepackage[latin9]{inputenc}

\makeatletter

%%%%%%%%%%%%%%%%%%%%%%%%%%%%%% LyX specific LaTeX commands.
\DeclareRobustCommand{\greektext}{%
  \fontencoding{LGR}\selectfont\def\encodingdefault{LGR}}
\DeclareRobustCommand{\textgreek}[1]{\leavevmode{\greektext #1}}
\ProvideTextCommand{\~}{LGR}[1]{\char126#1}


\makeatother

\usepackage{babel}
\begin{document}
\title{Diffraction with a Single Slit}
\author{David Cai}
\maketitle

\section{Introduction}

Single slit diffraction demonstrates the wave-particle duality of
light. Depending on both the wavelength of the light and the width
of the slit, interference patterns with peaks of different amplitudes
are formed at angle $sin(\delta\theta)=\frac{\lambda}{d}$ apart,
where $d$ describes the width of the slit. Depending on $\lambda$
and $d$, small angle approximation might be used. 

\section{Materials and methods }

\subsection{Brief description}

In general He-NE laser with a narrow waveband 632.8 nm is used to
approximate a monochromatic source. Ideally, a prism or a spatial
filter is used in order to reduce the aberration of the source. We
omit the source cleaning procedure in this lab since the quality of
our light source is good. The laser beam passes through the first
filter to coarsely filter the source. The light then passes through
another filter with narrower width such that we get our final diffraction
pattern on the viewing screen or the photon detector. The viewing
screen or the photo detector is installed on the right-end of the
table such that the diffracted angle can be measured more precisely,
under the assumption that our source is a monochromatic source. \textbf{Verify
that whether we do need the first slit or not.}

\subsection{Equipment }

HeNe laser; calibration tool; optical breadboard Table; coarse slit
on a film; a finer slit (around 100 \textgreek{m}m); viewing screen;
photo-detector with a coarse slit installed in front; a ruler is also
required to measure the distance between the slit and the measurement
device.

\subsection{Setup }

We utilize the length of the table by installing the laser on the
far left-end of the table. Calibration of the laser then need to be
done both vertically and then horizontally, according to laser's manual,
or use the ThorLab alignment tool. First slit should be installed
perpendicular to the light path around 10cm away from the source.
The second slit should then follow closely at around 3-5cm away. Place
the viewing screen on the far right end of the table such that the
angle difference can be measured as precisely as possible.

\section{Error Analysis}

The uncertainties come from the width of laser, the width of the slit,
distance between the slit and the detector/ viewing screen, the transverse
length the detector has moved, 

Data Analysis Bandwidth theorem is used, referring to Lab Manual of
optical pumping experiment. We combine this into our calculation. 

\section{Data Analysis}

The data we have post-experiemnt are the $l$, the distance between
slit and the screen, $d$, the width of the slit, $\lambda$, the
wavelength of laser, and a serie of transverse displacement $\delta d_{i},$where
the peaks interference are measured. For the photo-detector, we can
measure more data such that we can generate a continuous plot with
peaks and troughs at various angles. The purpose of generating this
plot is to compare with the plot 

\section{Goals}
\begin{enumerate}
\item Verify that the wavelength $\lambda$ is indeed 632.8 nm.
\item Verify that the amplitude equation (integral form) based on Fresnel
equation agrees with our finding.
\end{enumerate}

\end{document}
